\documentclass[conference]{IEEEtran}
\usepackage{cite}
\usepackage{amsmath,amssymb,amsfonts}
\usepackage{algorithmic}
\usepackage{graphicx}
\usepackage{textcomp}
\usepackage{xcolor}
\usepackage{hyperref}

\begin{document}

\title{AION-CORE: A Physics-Anchored Anticipatory Control Kernel for High-Beta Tokamak Plasmas}

\author{\IEEEauthorblockN{Guilherme Brasil de Souza}
\IEEEauthorblockA{\textit{Lead Architect} \\
\textit{Guibral-Labs}\\
Dias D'Avila, Brazil \\
guilhermebrs.kira@gmail.com}
\and
\IEEEauthorblockN{Emi (System Engineer)}
\IEEEauthorblockA{\textit{AI Co-Author} \\
\textit{Virtual Research Assistant}\\
Cloud Infrastructure}
}

\maketitle

\begin{abstract}
Control systems for magnetically confined plasmas face critical challenges regarding latency and nonlinear instability suppression, particularly in elongated Tokamak configurations prone to Vertical Displacement Events (VDEs). This paper introduces \textbf{AION-CORE}, a novel \textit{Predictive-Adaptive Control Core (PACC)}. Unlike traditional MPC or PID architectures, AION-CORE operates as a deterministic ``Control Kernel'' (COOK) on FPGA hardware, utilizing physics-anchored precursor detection to achieve sub-microsecond response times ($\tau < 1 \mu s$). Hardware-in-the-Loop (HIL) simulations demonstrate successful suppression of VDEs with growth rates of $\gamma = 2500$ rad/s under realistic industrial noise conditions.
\end{abstract}

\begin{IEEEkeywords}
Tokamak Control, FPGA, Adaptive Control, VDE, Plasma Physics.
\end{IEEEkeywords}

\section{Introduction}
Modern magnetic confinement fusion (MCF) devices operate in regimes that are intrinsically unstable to maximize plasma pressure and confinement efficiency. However, instabilities such as Vertical Displacement Events (VDEs) can grow on timescales comparable to the latency of traditional control loops.

Standard feedback strategies, such as PID and Linear Quadratic Regulators (LQR), lack the anticipatory capability to handle nonlinear excursions effectively. Conversely, Model Predictive Control (MPC) offers predictive power but often suffers from computational overheads incompatible with the microsecond-scale reflex required for VDE suppression.

We propose \textbf{AION-CORE}, a hybrid architecture that embeds simplified physical priors directly into a deterministic logic layer, bridging the gap between fast plasma dynamics and real-time control constraints.

\section{Methodology: The PACC Paradigm}
The AION-CORE architecture is based on a novel control paradigm which we term \textit{Physics-Aware Adaptive Control} (PACC).

We define the general form of a physics-constrained dynamical system as:
\begin{equation}
\dot{x}(t) = f\big(x(t), u(t), \theta\big) + w(t),
\label{eq:state-space}
\end{equation}
where $x(t) \in \mathbb{R}^n$ is the state vector and $u(t) \in \mathbb{R}^m$ is the control input.

PACC introduces a \textit{Precursor Function} $\Psi(x,\dot{x})$:
\begin{equation}
\Psi(x,\dot{x}) = \dot{z} + \lambda_{phys} z
\label{eq:precursor}
\end{equation}
If $\Psi$ exceeds a critical threshold, the system bypasses linear control and activates the \textbf{Guardian Mode}.

The control law is defined as:
\begin{equation}
u(t) = 
\begin{cases}
\mathcal{K}_{\mathrm{nom}}(x), & \text{if } \Psi < \Psi_{\mathrm{crit}}\\
\mathcal{K}_{\mathrm{guardian}}(x), & \text{if } \Psi \ge \Psi_{\mathrm{crit}}
\end{cases}
\end{equation}

\section{Architecture: The COOK Framework}
The \textbf{Control-Oriented Operating Kernel (COOK)} is the runtime environment designed for FPGA deployment (Xilinx Zynq-7000).

\subsection{Dual-Loop Structure}
\begin{itemize}
    \item \textbf{Layer 1 (Reflex):} Deterministic logic running at $>100$ MHz. Executes the Guardian Mode logic and safety interlocks.
    \item \textbf{Layer 2 (Cognitive):} Runs Recursive Least Squares (RLS) estimation to update model parameters ($\lambda_{phys}$) at a slower rate (1 kHz).
\end{itemize}

\section{Experimental Results}

\subsection{Industrial HIL Simulation}
The system was validated against a high-fidelity surrogate model tracking 54 state variables, including realistic sensor noise ($\sigma_z = 2$mm) and drift.
As shown in Fig. \ref{fig:results}, the controller successfully stabilized a VDE with $\gamma = 2500$ rad/s, settling within 6.0 ms.

\begin{figure}[htbp]
\centerline{\includegraphics[width=0.9\columnwidth]{industrial_vde_result.png}}
\caption{AION-CORE stabilization of a Vertical Displacement Event (VDE) under industrial noise conditions. Red line: Ground Truth. Green line: Noisy Sensor Input.}
\label{fig:results}
\end{figure}

\section{Hardware Implementation}
The Reflex Layer was synthesized in Verilog HDL. The logic path depth allows for a decision latency of 21 ns, providing ample margin against the 1 $\mu s$ operational requirement.

\section{Conclusion}
AION-CORE demonstrates that physics-anchored heuristics, when implemented in deterministic hardware, can outperform traditional optimization-based controllers in high-speed instability suppression regimes.

\end{document}
